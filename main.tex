% PRÉAMBULE
\documentclass[letterpaper, french, 12pt]{article} % classe du doc
\usepackage[french]{babel}
\usepackage[T1]{fontenc}
\usepackage{amsmath}
\usepackage{amssymb}
\usepackage{mathtools}
\usepackage{fullpage}

% Quelques options
\setlength{\parindent}{0 cm}
\setlength{\parskip}{\baselineskip}
\numberwithin{equation}{section}
\numberwithin{figure}{section}
% \pagenumbering{gobble}

% DÉBUT DU DOCUMENT
\begin{document} % environnement document
Ceci est un texte d'introduction à LaTeX.
Qu'arrive-t-il quand on saute une ligne?\\
Saut de ligne manuel

Ceci est un deuxième paragraphe.

Quelques commandes de base:

\section{Première section} % début d'une section numérotée
\section*{Section sans numéro} % On utilise une étoile pour dénoter une section non numérotée

\subsection{Ceci est une sous-section}
Ceci est le texte de la première sous-section.

\subsection*{Sous-section non numérotée}

\subsubsection{Sous-sous-section}
\subsubsection*{Sous-sous-section non numérotée}

\section{Commandes de format de texte}
\subsection{Pour faire des listes:}

1. Bullet points
\begin{itemize}
    \item Premier point.
    \item Deuxième point.
    \item Troisième point.
\end{itemize}

2. Liste numérotée
\begin{enumerate}
    \item Titre: Numéro 1
    \item Numéro 2
    \item Numéro 3
\end{enumerate}

\subsection{Format du texte}
\begin{itemize}
    \item Texte en italique: \textit{Italique}
    \item Texte en gras: \textbf{bold font}
    \item Texte sous ligné \underline{sous-ligné}
\end{itemize}

\section{Mathématiques de base}
Pour insérer du texte mathématique au milieu
d'une phrase, on utilise le signe de dollars:
\$...\$. Par exemple, soient $x, a$ et $b$, des entiers: $f(x) = ax + b$.

Pour insérer des mathématiques sur leur propre ligne, on utilise les brackets:
\[
f(x) = ax^2 + bx + c
\]

Il est recommandé par contre d'utiliser plutôt l'environnement "equation", car celui-ci est plus facile à lire et plus puissant. Cette commande numérote aussi les équations de votre document.

\begin{equation}
    f(x) = ax^3 + bx^2 + cx + d
\end{equation}

On peut aussi écrire une équation non numérotée avec "equation*" (requière un package par contre).

\begin{equation*}
    f'(x) = a_0 + a_1x + a_2x^2 + \dots + a_nx^n
\end{equation*}

\subsection{Quelques caractères mathématiques de base}
On a:
\begin{itemize}
    \item Les opérateurs de base + - * / :
    $a + b - c * d / e$. On peut aussi utiliser
    "times" pour imprier le 'x' de multiplication
    classique: $x \times y$. Sinon la multiplication est explicite entre variables: $ax + by + c$
    \item Les opérateurs exposant et indice: \^\ et \_ : $a_0x + a_1x^2 + x_1^3$.\\
    Par défaut les indices et exposants prennent
    un seul caractère, mais on peut utiliser
    plusieurs caractères en accolades: $e^{2x+1}$
    \item Pour la division, on va souvent vouloir
    utiliser l'opérateur "frac", qui permet de faire des fractions plus esthétiques:
    $\frac{1}{2}(ax + b)$, ou bien $\frac{x^2 + 1}{x^3 - 1}$
    Noter que certains caractères mathématiques sont plus difficiles à lire quand ils sont
    sur la même ligne que le texte. On préfère alors l'environnement équation~:
    \begin{equation}
        \frac{x^2 + 1}{x^3 - 1}
    \end{equation}
    \begin{equation}
        \sum_{i=1}^{n}x_i
    \end{equation}
    Somme enligée avec le texte: $\sum_{i=1}^{n}x_i$.
    On peut aussi utiliser l'exposant et l'indice
    pour faire des fractions: $^3/_4$. Il existe 
    des packages qui mettent cette notation dans une commande (ex. xfrac).
    \item Les inéquations: $< \ \leq \ > \ \geq \  \neq \ \equiv$.
    \item Caractères propositionnels: $\forall \ \in \ \exists \ \subset \ \subseteq \ \supset \ \supseteq$
    \item Les fonctions de base: $\sin(\theta), \cos(x), \tan(y), \arcsin(z), \sin^{-1}(x)$
    \item Si on besoin d'écrire du texte dans les mathématiques, on utilise "mathrm", mais souvent "text" est préféré (fournie par le package amsmath): $\mathrm{Im}(A)$ On a vs On a $a$.
    \\ On préfère plutôt "text"
    par contre: $\{x \in V \ | \ x \text{ pair et } x < 3 \}$
    \item \[\int_a^b x \ dx\]
    \[\iint_a^b\]
    \[ \sum_{i=0}^n x_n \]
    \[  \prod_{i=0}^n x_n \]
    \[ \lim_{n \rightarrow \infty} \]
    \item Lettres grecques: $\alpha, \pi, \beta$
    \item mod et binôme: $a \bmod 5$
    %\[ \binom{k}{n} \]
\end{itemize}

\section{Petit détour sur les packages}
Pour importer un package, on utilise la commande "usepackage"

Quelques packages que je recommande d'utiliser:
\begin{itemize}
    \item amsmath: Rajoute un paquet de fonctionnalité aux maths dans LaTeX
    \item amssymb: Rajoute un paquet de symboles manquants.
    \item mathtools: Rajoute des
    utilitaires pour les mathématiques
    \item babel: Sert à localiser la langue du document (ex. mettre en français). On voudra alors
    aussi utiliser [T1]fontenc
    pour que LaTeX supporte tous les caractères français.
\end{itemize}
On a aussi quelques packages et options qu'on peut utiliser pour modifier le comportement de LaTeX et créer un document qui fait plus "devoir de math".
\begin{itemize}
    \item Options:\\
    $\backslash$setlength\{$\backslash$parindent\}\{0 cm\}\\
    $\backslash$setlength\{$\backslash$parskip\}\{$\backslash$baselineskip\}
    \\
    $\backslash$numberwithin\{equation\}\{section\}\\
    $\backslash$numberwithin\{figure\}\{section\}
    
    \item packages:\\
    fullpage
    
\subsection{Fonctionnalités avancées mathématiques}
Grâce au package amssymb, nous avons maintenant accès aux lettres doubles pour les ensembles habituels: $\mathbb{R}, \mathbb{Q}, \mathbb{N}$

On a aussi maintenant accès à quelques environnements supplémentaires pour les équations.

\subsubsection{L'environnement align}
Permet de combiner plusieurs équations dans un même bloc et de les aligner à l'aide de marqueurs.
\begin{align}
    \sum_{k=1}^{\infty} \frac{1}{k} &= \frac{1}{1} + \frac{1}{2} + \frac{1}{3} + \dots + \frac{1}{n}\\
    &= \left\{\frac{1}{1} + \frac{1}{3}\right\}
    + \left(\frac{1}{2} +
    \frac{1}{4}\right)
\end{align}

\begin{equation}
    \begin{split}
        1 + 2 + 3 + 4 + 2 + 3 + 4 + 5 + 1 \\
        + 2 + 3 + 4 + 1 + 1 + 2 + 3 + 4 + 2 + 3 + 4 + 5\\
        + 1 + 2 + 3 + 4 + 1 + 1 + 2 + 3 + 4 \\
        + 2 + 3 + 4 + 5 + 1 + 2 + 3 + 4 + 1
    \end{split}
\end{equation}

\begin{equation}
    e^{-\log(1-u)/\alpha}
\end{equation}

\end{itemize}

\end{document} % fin env. document
% FIN DU DOCUMENT